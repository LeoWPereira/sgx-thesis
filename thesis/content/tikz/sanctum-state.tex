\begin{tikzpicture}[
	state/.style={draw, ellipse, text width=1.5cm, minimum height=1cm, align=center},
	arrow/.style={-latex'},
	heading/.style={font=\bfseries},
	label/.style={text width=3cm, align=center}
]
% enclave state
\node[heading] at (5,-2) {a) Enclave state diagram};

\node[state] (v1) at (0,0) {non-\\existent};
\node[state] (v2) at (5,0) {loading};
\node[state] (v3) at (10,0) {initialized};

\draw[arrow]  (v1) edge node[above, label] {create enclave} (v2);
\draw[arrow]  (v2) edge node[above, label] {init enclave} (v3);
\draw[arrow]  (v2) edge[loop, looseness=5] node[above, label] {load page} (v2);
\draw[arrow]  (v3) edge[loop, looseness=5] node[above, label] {enter enclave} (v3);
\draw[arrow]  (v3) edge[bend left=15] node[below, label] {delete enclave} (v1);

\node[state] (v1a) at (0,-5) {free};
\node[state] (v2a) at (5,-5) {initialized};
\node[state] (v3a) at (10,-5) {running};

% thread state
\node[heading] at (5,-7.5) {b) Enclave thread state diagram};

\draw[arrow]  (v1a) edge[bend left]  node[above, label] {load thread} (v2a);
\draw[arrow]  (v1a) edge             node[above, label] {create thread} (v2a);
\draw[arrow]  (v2a) edge[bend left] node[below, label] {delete thread} (v1a);

\draw[arrow]  (v3a) edge[bend right]  node[above, label] {exit enclave} (v2a);
\draw[arrow]  (v2a) edge             node[above, label] {enter enclave} (v3a);
\draw[arrow]  (v3a) edge[bend left] node[below, label] {asynchronous enclave exit (AEX)} (v2a);

\draw[arrow]  (v3a) edge[loop, looseness=5] node[above, label] {resume thread} (v3a);

\draw[dotted]  (-1.5,-2.5) -- (11.5,-2.5);
\end{tikzpicture}