% => Wenn die Arbeit auf Deutsch verfasst wurde, verlangt das Studienreferat KEINEN englischen Abstract

% % englischer Abstract
%
\begin{otherlanguage}{english}
\begin{center}\textsf{\textbf{Abstract}}\end{center}

\noindent
The release of Intel SGX revived the interest in trusted computing across industry and academia.
Hardware is available, but usage patterns and applications are mostly missing.
This thesis evaluates trusted computing from the viewpoint of a software engineer.
Hardening strategies are identified in related work and applied in two case studies.
The case studies show how SGX can be used in practice. A small helper library is developed for rapid prototyping.
Other trusted computing solutions are compared to SGX and SGX is critically evaluated based on current research.
\end{otherlanguage}

\vspace{2cm}


% => Wenn die Arbeit auf Englisch verfasst wurde, verlangt das Studienreferat einen englischen UND deutschen Abstract

% deutsche Zusammenfassung
\begin{otherlanguage}{ngerman}
\begin{center}\textsf{\textbf{Zusammenfassung}}\end{center}

\noindent
Die Veröffentlichung von Intel SGX hat das Interesse an Trusted Computing in Akademia und Industrie wieder erweckt.
Obwohl die Hardware verfügbar ist, sind Verwendungsmuster and Anwendungen noch Mangelware.
Diese Arbeit evaluiert Trusted Computing aus der Sicht eines Software Ingenieurs.
In verwandten Arbeiten werden Strategien zum Härten von Anwendungen identifiziert und in zwei Fallstudien angewandt.
Diese Fallstudien zeigen wie SGX praktisch genutzt werden kann. Dabei wird eine kleine Hilfs-Bibliothek entwickelt.
Andere Trusted Computing Lösungen werden mit SGX verglichen und SGX wird mit Hilfe aktueller Forschungsergebnisse kritisch bewertet.
\end{otherlanguage}



